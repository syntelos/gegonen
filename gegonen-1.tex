\input gegonen

\centerline{\bf Gegonen I}

\smallskip

\centerline{John Douglas Harold Pritchard}

\break


Gerard and Sophie took a walk down Broadway.  The cold air was colder
under the Flatiron.  Far from home.



Gerard picked up a wifi on a closed store.  ``Got one.''

Sophie looked at the map.  ``Sure''.

Gerard rewrote the firmware on the router for public access.

Sophie checked the bank.  ``Um.''



Gerard and Sophie walked down Lexington into Foggy Bottom where
Sophie's wave rider was sitting on their pylon.  The community mesh
net was done for the night, which had set in firmly.  Another pylon
had gone up since leaving the ship that morning.

They ascended a spiral stair to the air lock.  It opened with warmth
and light and all of the promises of a home in the sky.  The plan was
to sleep.  The ground water cistern under the pylon was half full.
More than enough for a bath.

Gerard took another look at a lunar flight plan.  He longed to return
to space.  Sophie saw what he was looking at.  She put her hands on
his shoulders and freed the perpetual bond across the back of his
shoulders.  ``Soon'', she said.



As morning broke the eastern horizon from the anonymity of night, the
wave rider glowed like an opaque milky quartz teaming with strands of
micah and pyrite.  Sophie turned on her side to look out through a
transparent slice in the wall.  It was no mistake that her ship was
colored like the horizon over Beaucatcher Mountain right now.  The
micah and quartz in the ground matched the colors in the oceanic
skies, here.  It was foreign to Luna, her orbital perspective, but
then it was always home.  A return to lunar orbit was nearby.  A good
idea.



Sophie's thoughts about a return to the moon had shifted overnight.
Time spent in her wave rider?  So it would seem.  Another mention and
it had tipped.  The world would live without them.  The bank was
replenished, and the ship was on budget for a flight.  They had no
where else they would rather be.  Solitude, for a spell.



Gerard saw Sophie staring out the window and wondered if she was
thinking about the moon.  There was a moment, there, that seemed to be
arriving.  They could and they should return to the moon.  It was so
peaceful, there, in solitude.  He could stay longer than Sophie's wave
rider could, and thought of a larger ship.  The work sheets for
budgets he produced said a couple things to him.  First that he would
need a more substantial income stream, and second that the size class
proposed a prohibitive parking arrangement.  It wasn't obvious how
four pylons for one ship was a public good.  He thought it over again.
That he believed the supporting argument to be the arbitrary
difference between one or two or three or four pylons.  The pylon grid
on equilateral triangles provided space and convenience and
established the ``one'', ``two'', ``three'', or ``four'' gridding
classses.  But building and keeping a ``four'' class ship in Foggy
Bottom would raise plenty of discussion.  The counter claim held a
unit's privacy and a share's economy.  Taking multiple spaces long
term would cost a share of the ship, credited to the community in some
sense.  His income plan was a community contribution, but as it
produced the income to support the ship, the application of the
benefit to closing the share would appear to be double counting the
thing.  Making distinct the community cost and benefits of pylon space
to voices on behalf of relatively aggressive precedent and practice
for sharing arrangements would be a matter of politics.  His living
arrangements were not an issue.  The allocation of pylons were an
issue.



Sophie rose first.  She went to the galley and prepared coffee.  She
enjooyed lifting him with the smell of a new morning.  Gerard was not
a morning person.  He drifted across the floor from the aft bunk.  He
found the back of her neck with his head, and managed to plant one on
target.  His arms wrapped around her.  She held his embrace to herself
and closed her eyes.





The coffee hit him like a ball.  His ball of his morning.  The smell
in his nose, and the simplicity of the tea of roasted beans in his
mouth woke him.  Such a habit.  That she should employ such devices to
reach him a condition of his inability to put the march aside.  She
liked the march.  It was her march.  She would march him to the moon
over and over for the rest of their days.  It was their life.

Sophie looked into Gerard's eyes, ``Awake, yet?''.  Gerard raised a
pair of eyelids to her.  ``Not yet.'', she said to them.  For them.
The actual march, occurring to him as she slid breakfast in front of
him.



``You could return to the university and accomplish either budget.''.
Sophie referred to Gerard's original career as researcher and
lecturer.  The symposium for special research had bumped his field
from the decadal survey, putting his lectures and budgets in the wash.
Optional.  Artificial orbital science was a special subjects
classification under the space research agency.  It had become a minor
activity during the reconstruction of the radio telescope on Battery
Hill, and would not become available again for more than a decade.  In
the meantime the best place for Gerard was lunar orbit.



Unlike Gerard, Sophie was entirely content to live on the fruits of
her labors.  She went to school to complete her childhood interest in
Molecular Fabrication.  In return, she contributed to the community
periodically with developments in Instrumentation and Fabrication.
She had repaid the gift of education many times over, and continued to
give everything she had.

Gerard's interests took him to another stage, one that depended on
facilities he could not always realize single handedly.  A radio
telescope or an orbit.  That Sophie should be so adept with orbits was
another of a million reasons he had for his being (perhaps) a bit too
dedicated to her.



Sophie used the artificial orbital science data system with the
orbital navigation scenario planning system in her wave rider to solve
Gerard's network problem.  His dataset accumulated network modeling
output into unit automaton credit and deficit figures.  Sophie came up
with a spectrum of possibilities which she plotted over a manifold of
lunar time on orbit, and network benefit.  She gave Gerard a contour
map with a pair of lumps for vacation time.  Either orbit had its
blessings, and maximizing timee in one rather than cislunar space was
a spiritual preference that they shared.  



Work in Gerard's field supported the implementation of objectives from
other fields including planetary science, space flight science, and
stellar science.  For Gerard and the members of his community,
everything fell into one of these bins.  The planetary and planetoid
orbiters from the academy of fine arts were planetary science automata
from the perspective of the artificial orbital science community.  The
asteroid mining and communications infrastructure were stellar
science.  Gerard's interest was in delivering automatic systems for
the best realization and maintainence of those projects.  When a
science project expired, if it did, the implementation came under the
domain of the artificial orbital science community.  The abandoned
material was supported, repurposed, recycled, or disposed of as they
found prudent, necessary and expedient.  These tasks were handled by
public committees and their public participants, who proceeded through
the production and publication of letters and votes with each step of
each subject under their purview.  

With a radio telescope, terrestrial command and control was
nonexistent.  It was not, strictly speaking, necessary to the
operation of the perpetual stream systems in space.  The extent of the
deep space network would expand according to the progression of
centuries to provide its next milestone in extra solar bandwidth
facility.  

Interplanetary internet would need to be fit to each task, a far more
complex decision tree for the committees of the artificial orbital
science community to navigate.  The budget for these activities would
be displaced by the budgets for new activities while the radio
telescope was offline.  An admirable trait of his fellow human beings,
while entirely inconvenient to the members of his field -- excepting
himself.  Gerard and Sophie could repair and replace near field
interplanetary internet automata, and in a sense they were a program
counter on the community's cost and benefit computer.



The construction network, STC, mined the asteroids for material and
the system for energy in the objective program to deploy a star scale
communications network.  The sphere of STC nodes at one light year
diameter repeated signals captured through long and very long baseline
interferometry over its surface.  The STC maintained repeater rings
throughout Sol, and included the AFA/PN among its own without
distinction external to the ownership and access bags.  The signals
economy in artificially occupied space provided available bandwidth
over distance to the cosmos.  The ethereal advance first most sensible
to the physical economy of earth.



A construction automaton had network interfaces by Gerard and a number
of necessities of structure and architecture by Sophie.  The
construction automaton pulled materials from the asteroid belt and
flew it into orbits established and updated by the AOS STCC.  The AOS
STCC was a closed self organized group.  As the AOS formed around the
facility and capacity to realize generalized artificial space flight
in self reproducing automata, a governance committee formed to
delegate topics to its charters.  This behavior on behalf of the
scientific community reflected the fact that collaboration from the
center of the field would be necessary to the development of molecular
fabrication technologies.  This perspective was widely recognized as
natural and sensible, trustworthy and worthwhile, and therefore the
natural course of self government for individual and society alike.

The construction network was slow, its strengths in communication
limited by distance and leveraged over its delivery of perpetual
income streams.  This wealth was allocated by the construction system
that other members of the community developed.  It maintained an
ecosystem that fed into the economy of the Sol system at established
devices of necessity of government.  The artificial orbital community
allocated resources to itself in the development of the Sol system for
the long term -- as afforded by the situation over term.  Inwardly AOS
were strategy poor.  Interplanetary comnunications support was
produced by other sectors of society for their own needs and
objectives.  Interplanetary internetworking for the AOSC could be
rather limited while the STC was advancing outward.  It was a truth
that told the story of the public good to be derived from the STC,
about the public good derived from space and its longer points of
concern, but left Sophie and Gerard to fend for themselves in their
individual existence and to repair, recycle, repurpose and develop
resources accumulated by the AOSC for their interplanetary and
interpersonal internetworking.



They could leave tomorrow.  Gerard mentioned visiting the AFA with the
plan as an excuse to see Carl.  Sophie mentioned the flight to Teresa
and met her at her place for lunch.  So perhaps not tomorrow.



Between the dry ridges that contained the Catawba Turnpike -- Broadway
between Greenville, Tennessee and Greenville, South Carolina -- at
Asheville was Foggy Bottom.  Over the western ridge was a drop down to
Lake Pisgah.  Under a cold morning sun in January the lake was frozen.
The Smokey Mountains covered in snow.  The chalet of government
overlooked the heights from the north by north east.  To its southern
exposure the sun rise, city and lake.  Beyond the lake over the far
horizon Mount Pisgah, Little Pisgah, and Cold Mountain.  All the earth
within the horizons, here.



Carl walked home over the lake thinking of Gerard.  It was always
Gerard who would put a couple stitches in the ways and means to keep
things moving.  He and Sophie a special miracle among the heros who
lent their support to AOS.  The appreciation of life was heightened by
the steady and reliable dependence on the AOSC at large.  That one of
their own -- known in person -- should perform this vital r\^ole a
pleasure to participate in.  Carl was an aquanaut, himself.  He
enjoyed the life under the surface of the lake -- in West Asheville,
as he called it.



Under the lake a transparent home enjoyed only as much light as the
depths afforded.  Carl's home had two stories above the surface and
more than ten stories below.  His plan was hydrothermal.  Construction
would halt in another decade.  He had employed wind and solar with
fusion to produce the energy budget required to support his project.
There was more than enough space for the structure to take on a range
of possible r\^oles, private or institutional, when he n longer
required it for himself.  But there was little budget in the plan he
elected for vehicles.  He kept a simple electric aircraft for his
basic freedom, aside from his legs and arms, a kayak, a canoe, a
sailboat, and a bicycle.  The bike was kept in town, and the others
were stored in the enclosure that he called a boathouse.

The first floors above and below were organized as kitchens.  He
simply couldn't pick a favorite.  He would use either one according to
the light and his mood.  Likewise the second floors above and below
were first bedrooms for him.  He didn't always enjoy sleeping under
the stars, in the atmosphere over the lake.  He sometimes slept well
under the surface, only close enough for a comfortable awareness of
the world outside.  This calm was for him what he understood from
Gerard's description of his interest in lunar orbit.  A quietude that
instilled serenity, as surely as opening a door and walking into a
room.



In the last hours before sunrise Gerard and Sophie were speaking of
their hearts' desires.  To return to space flight with the plan that
accomplished both a return to the moon and the best they could do for
AOS resources.  They would rendezvous with two birds in low earth
orbit, make the big swing through cislunar space, and then meet each
of the three AOS birds in lunar orbit that required periodic
refueling.  Each bird would get the full attention of an automatic
mechanic that would trade its time and materials and energy budget for
the health and well being of the subject automaton.  Any depletable
resources would be replenished, damage and fatigue would be repaired,
and then engineering requests for modifications and updates would be
processed according to opportunity.  The schedule of events derived
from the flight scenario would deliver one hundred percent successs
for each of the five rendezvous.  This contribution was not a
hardship, it was all fairly easily done with conventional technology,
but nonetheless it filled them with the joy of accomplishment that
they knew would be shared by everyone concerned with the ways and
means of doing the work of artificial orbital science.  Communications
to the array of sensors and systems on the lunar far side would be
improved by two hundred percent, and then the effectiveness of the AOS
presence in LEO would be improved by ten percent.  The moon was an
obscure point for many, and the center of the universe for some.  The
radio quiet and mass and proximity of the far side location made it
practical for many objectives.  The vary large radio telescope
measured tens of kilometers in diameter, was etched into the surface,
and could be seen from orbit.  The AOS provided communications links
for principal research access.  That it was very nearly offline,
otherwise, a source of great concern for those fields dependent on the
data it provided.



When Sophie asked Gerard on their first date, he was a bit surprised.
He was lecturing a graduate level course on automata theory, and she
was a graduate student in physics.  She had no reason to expect that
she would accept.  Quite boldly she entered his office during the
period allocated to his students to follow up on subject material,
coursework, and related housekeeping activities.  Occasionally a
student would want to learn about his doctoral thesis, a book or paper
he'd written, or some AOS activity.  No one had ever asked him a
personal question, and at the time his personal life was generally
quiet.  The occasional meal or party with friends, especially those
events marking AOS accomplishments.  When she stepped into the room,
he recognized a graduate student of physics as one has interest in the
future.  She stood him up when she leveled a look at him and proposed
dinner.  A romantic interest.  An experiment in love binding.  He
thought for a moment, standing in front of her.  He knew how to search
his feelings.  He knew he was less than ten years older than her.
While sudden and unexpected, not unreasonable.  And with as much as a
comfortable feeling of warmth from within he accepted.  It was not the
jump off a cliff impetus that some had spoken of, but there was no
guarantee that that experience was universal.  As they discussed that
first plan he became aware of her certitude.  She was a graduate
student of physics and adept with certainty and objectivity and
subjectivity.  Her confidence opened him up like a caught fish
prepared for the pan.  And it hit him.  He flushed and took his seat.
She saw through her own hazy romantic consciousness that he had gone
blotto.  She took her seat and fell silent with the victory.  He took
a moment to find a handle.  To realize the truth of what had occurred.
He had just fallen in love.



To her surprise, Gerard broke that first silence.  Never since has she
wanted for a word or gesture from him.  He simply told her that he had
fallen in love with her, in that moment, and that his feelings were
very strong.  She could rethink her proposal if this information did
not meet her needs.  She smiled in beams of glory and told him that
she would have to kiss him if he spoke another word.  He took a moment
for the emotional event to collapse into some form of consciousness
and said ``yes''.  She planted a big one on his lips and ran out into
the hallway, stopped to catch her breath, and walked away.  He
collapsed into his chair.  The next thought to enter his mind was the
place and time of their dinner date.  Tomorrow night.  It would be an
eternity.  He would be checking his messages twice as frequently in
case she sent him one.  She knew his name and address, by office, but
he didn't yet know her name.



At the time Gerard slept his first night in Sophie's absence he was
living in an apartment allocated to faculty.  He wondered what her
name might be.  He agonized over the seconds and minutes occupied by
thoughts of her.  The promise of a future mated in couple didn't occur
to him as such.  A young man is excited by the promise of sexual
activity.  And while that sugar was rotting his brain, his heart sang
the tune that actually captivated him.  And for this reason he did not
reject the experience of imagining as shallow or not worthwhile.  He
knew that in that boiling cauldron of emotion were feelings of import
and substance beynd the expression of mating by the act of
reproduction.  Not knowing the art of life in couple, he had only a
subliminal appreciation for the song his inner consciousness was
singing to him.  He might mistake instinct for admiration or {\it vice
versa}, but he could not mistake the commitment his heart's song was
preparing him for, and the fact that this song and his experience of
it brought him enormous wealth of spirit.  He thought of fatherhood
and spousehood.  Some may think it out of ordinary, but it was
spousehood that drew him into the spiritual heights.  And then he
understood, in some sense, that the heights to which he aspired had no
relation to sexual intercourse.  That they were spiritual, and that
they were about her.  He felt the notion of a purpose or calling, and
its name was husband.



Sophie returned to her apartment full of her respect and admiration
and now full bloom love for him.  She did not think of children.  She
thought of the feelings she had for a mate, the promise of a couple,
and wondered again at the mystery of this life that we are so drawn to
each other as to love and fight continuously if not for the maturities
of self awareness.  If fighting and conflict are ultimately
expressions of ignorance, and love an expression of our genetic
origins and evolutions, then could we not escape love as we endeavor
to escape conflict in order to move ourselves forward through time.
At least that's a hyper-rational graduate student perspective.  Not
satisfying.  Why would I try to oppose what I am when the problem I'm
trying to solve is how to get more work done.  Sophie felt relieved at
the reception she enjoyed with Gerard, but lived in another situation.
She was still climbing the mountain that Gerard had climbed.  She
struggled with the intellectual muscle as all students do.  She knew
that she was going through an experience of growth that would deliver
her to the possession of strength.  That had nothing to do with her
feelings for Gerard.  She had seen him, had gained interest and found
out who he was, and then eventually her interest grew to feelings of
admiration and the precipace of love that she had to face before she
went mad with it.  She had tried to avoid it, to put it aside for a
more convenient moment, but the feelings ate at her ability to get
work done.  Appropriately enough, from her perspective, she went to
see him during his office hours to place her bet and get a response.
To realize the problem and in this way to deal with it.  To go through
it rather than around or over or under it.  The success she met with
blew open a hole in her breast and let it out.  It was now a whole new
problem, but one that Gerard would understand.  She found the courage
or the will to put together something to eat and to get enough work
done that she could retire to sleep.



Sophie woke to a swarm of flying lights modeled after bumblebees, the
result of her master's thesis in molecular fabrication: the
reproduction of organic mechanical techniques involving flight at low
reynold's numbers.  She had designed a particularly complex and
challenging thesis project that covered a number of relatively novel
materials and compositions with a demonstration objective that
declared a certain passion for the work.  The bees collected energy
from the environment and had a flight ceiling more than hundreds of
meters.  The microcosm of the project had fascinated Sophie, although
it was not as exciting as an ultralight that could fly to space and
return to earth in an organic cycle.  That was the primary objective
of her doctoral thesis.  If she couldn't complete that objective in
reasonable time, she would pick up the parts remaining and report on
that.  If she could complete it, her novel term ``organic cycle'' and
its definition excited her most.  This morning she woke to see some
new colors.  She had built them with encephalographic sensors
following some work she found on the subject.  It was just the kind of
tangent that made the work fun.  Her eyes focused on the colors and
remembered Gerard.  The lights told her conscious mind what her
subconscious mind was thinking.  She was in love.  The evolution in
the colors told her she was waking up, and so she got out of bed and
made some coffee.  Many hours to go before meeting Gerard at Lucia's
tonight.  She'd be back to shower before then, so there was no need to
think about what she would wear.  Anything would do.  She was
generally functional despite being more emotional than normal.  The
bumblebees were programmed to not go out the front door, and in her
absence to rest where they maximized energy collection.  She recalled
the plan to diversify their behavio, and the fact that that objective
had not yet reached the top of her priority stack.  She headed out the
door with everything she needed in a shoulder bag, and would not
return till dark.



Gerard woke to thoughts about the life of a graduate student, and how
his life was unlike the life of a graduate student.  In the five years
since he completed his thesis he had grown comfortable in the campaign
against his own ignorance.  He could consume a day of work without
notice.  A graduate student is still building the intellectual muscle,
still feeling the work and relatively buried under the effort.  In
this sense the relationship that promised to continue tonight would be
found on some uneven territory.  He was feeling old.  Older than her.
He was thirty and she would probably not yet be twenty five.  He
packed himself into some clothes and walked out the door wondering if
it would work, wondering if the differences between them would betray
his heart and cause him the legendary pain of divorce.  The shook off
the thought as fear of the unknown and self indulgence.  He didn't
need to dig that deep to see that he loved her, and to know that life
was as simple as that.



Much to his surprise, Gerard saw Sophie walking up to Lucia's as he
was walking up to Lucia's.  His chest rose with the anxiety of someone
afraid of risking the exposure of his heart.  His head flooded with
the confusion of someone unsure of his steps.  He watched a smile form
on her face and realized that his face wore an expression of some
concern.  With the awarenness coupled with her smile he relaxed and a
smile emerged.  It may or may not have said I love you, but that was
the extent of his semi-conscious awareness.  They stepped up to each
other before Lucia's door as the only two people in existence.  Gone
were their differences along with any sense of academic life.  Gerard
knew there would be a kiss in this moment somewhere, and prepared
himself to not be any more forceful or masculine than she required.
He watched her eyes look into his.  Her expression seemed to narrow to
an inspection before it was swept into the grace he recognized as
generosity and felt as the arrival of the kiss.  Seconds gave way to
tenths of seconds as he leaned his head to hers and waited for her
head to lean to his.  She moved into him with a force that took his
breath and they were kissing.  It was a simple kiss of meeting lips
but it lasted for thousands of milliseconds.  He lost track of time
and space entirely in the embrace that joined them.  He was not
conscious of it, but he lost track of himself within the presence of
theirself.



Sophie saw Gerard walking up to Lucia's and thought this is it, this
is how it ends.  Her sense of humor a bit morbid for the observer
really only made sense in her head.  She began to worry that he would
see the senselessness of her personal expressions and believe her
confused or weird or maladjusted.  This fear of rejection rose within
and made her legs feel weak and wobbly.  Somehow all of this made her
smile.  She thought herself funny with a constant periodicity.  The
look on his face changed from something grave and serious to something
lighthearted and open as her mood changed.  Yep, this is it.  This is
the one.  It did not matter that their actions and reactions may have
been normal or abnormal or clich\'e or refined, there was no world
beyond herself and Gerard.  She stopped in front of him.  The door to
Lucia's was nearby, but no one was coming or going.  The world could
wait for her to kiss her man again.  With the following breath, right
on cue, his head moved toward her and she was gone.  He's mine.  I'm
kissing him.  I'm kissing him, kissing him, kissing him till the cows
come home.  Oh, je t'adore.  Je t'aime.  Je suis libre.



When he had had enough of not looking into her eyes, Gerard pulled
back and opened his eyes.  Despite the late hour of the day his being
was puffy with the emergence of consciousness.  Through this fog he
found her eyes where he expected to find them, with her nose almost
touching his.  They both smiled simultaneously and his blood rose
within his self to a gleeful joy.  His smile broadened and with it his
heart.  He took her hand in his and her beaming smile filled his soul
till the forces within him forced the capitulation of a break in the
moment.  Let's get something to eat, he spoke.  Like a prophecy he
suddenly felt light-headed and depleted.  Her smile changed into an
acknowledgement and her hand squeezed his.  He turned toward the door
in a half step and she followed.  Another step and they were walking
to the door which he opened with his free hand.  As the door was
opening a rush of pride of presence took hold within.  He was so in
love with her and it appeared as an aura that enveloped the pair of
them.



Sophie felt Gerard pull away and opened her eyes.  She could have had
more of that, but was happy to see him.  She saw in his eyes and found
a reflection of her feeling of a lapse in time.  It could have been
ten seconds or ten minutes, she felt like she had no idea and
discarded the notion of trying to focus on how much time had elapsed.
She smiled and when she did she noticed that he had smiled at the same
time.  She felt like she could feel his heart beating, and realized it
was her heart beating in her chest.  He took her hand and her focus
rotated from her heart to his hand in hers.  She realized she was
smiling when he spoke of eating.  She was starving.  She squeezed his
hand thinking, let's eat!  As they turned and walked through the door
she felt something she had never felt before.  Togetherness.  That
together they were more than two individuals.  Or that together they
were something different than two individuals.  She felt that she had
changed.



Have you studied psychology?  Gerard asked Sophie.  He had her name,
and they were returning to a state of consciousness that facilitated
the satisfaction of the need to talk, to converse and in this way to
interact intellectually.  That next drop of the existence of Gerard
and Sophie.  They each recognized that this was already far beyond
getting a name and address.  He had no address for her and didn't
think of it.  A subconscious thought of contacting her in future was
displaced by a subliminal reaching for not letting her get away.  

Yes, some.  She replied.  I enjoy Gefou's conception of psychology as
representative of the acquisition of character independent of the
architecture of society that defines normal, and relegates the
remainder to anthropolgy.

Well, ok.  Gerard smiled.  He was struck.  Shot down like an amateur
caught in front of an ace.  Guess we're done with psychology.  And the
word comprehension?

Comprehension?

Same idea.  More fun as the acquisition of character that engages like
an envelope or container rather than a meeting or overwhelming.  

Ah.  Right.  Cool.  Sophie paused to reflect on the whole.  Wild.  

The fighter pilot analogy fit to the age of missles and jets when the
duration of the affair was constructed around the time of flight of a
supersonic missle.  The analogy failed at the fact that both had hit
their targets successfully.

They each took a sip of water and stepped away from the displeasures
of intellect.  Too objective.  Gerard saw the writing on the wall.

Sophie, can we stay together after this?  I have a problem letting you
go.

She smiled and replied.  Yes, me too.



You're not concerned with the development of the couple and the
malignment of the individual?  Sophie asked Gerard.

I'm betting that the destruction of psychology covers all that and
more.  Gerard replied.

So then we're building a couple as the balance of each self with the
joined self.  Sophie conjected.

Yes.  I think that a couple formed from two individuals who are
lifelong practitioners of self balance is incapable of the malignment
of either or both individuals for much more than a hundred
milliseconds.  Gerard countered.

Have you read much Gefou?  Sophie asked.

No.  I haven't really read any Gefou.  Just a vague awareness.  Gerard
replied.

Oh, no!  Well, I love Gefou.  I've read plenty of her work.  I
shouldn't say plenty.  I should say some.  But I find it
authoritative.  Sophie replied.

You're saying it concludes the metaphysical line of inquiry.  Gerard
posited.

Yea.  Her capture of the distinction between effect and utility almost
propelled me into a career in metaphysics.  I could throw those knives
all day long.  Sophie proposed.

Um.  Ok.  You lost me.  Gerard paused and reflected.  Utility and
effect of knowledge development over time scales long and short.  Got
it.  Gerard repled.

Oh, you are a quick one.  Sophie smiled.  Gerard, I'm done.  She
continued.



Gerard and Sophie finished dinner over stories of youth and life that
illuminated their respective experience and in this way shared pieces
of their respective existential situations.  Sophie had the more
interesting stories and Gerard the more interesting reflections in the
balance of their feelings about the moment.  That is to say that
Sophie had a more diverse experience of life in which her existential
situation was bed, and Gerard had lived more in his head and had had a
more private and introverted existence.  This difference caused him
some concern which he gave voice to.  You don't think our existential
differences are a future point of failure?  No, I like the difference.
I need the contrast.  The diversity.  The contribution.  She replied.
He was surprised, having expected a very different desire on behalf of
someone with a more textured experience.  I find it difficult to shake
the idea that you would not desire a broader experience of your
partner.  He said.  That's clich\'e, Gerard.  Technically an insult,
but I'm not insulted.  You're just letting me know where your
insecurities are, and that's what I really need.  She replied.  I need
a partner.  A confident and a lover.  I need the man I love.  She got
him over the hump of self doubt, and recognized that that would be a
recurring theme for a while as the found the common ground that they
would share.

Where are we going?  Sophie asked Gerard as they walked out of
Lucia's.  I have a full sized apartment that's fairly comfortable.
You?  He replied.  Nothing special.  Let's go to your place.  And what
that she had elected a new home.  Gerard and Sophie walked the
distance to Gerard's apartment and spoke of the needs to balance
couple and individual, in this way recognizing the journey they had
elected to embark upon.  Their habits of work were implicit, and
forgotten as a natural force that would continuously reinforce the
individual.  In this sense their conversation was immature or vain,
neither theit individual nor their couple would ever require conscious
support because their organic senses possessed every means of
maintaining their spiritual health.  They arrived at the apartment
exhausted from their discussions, pur fresh clothes on the bed and
feel asleep holding each other.  Sophie dreamed of her bumblebees and
rolled away from Gerard in her sleep.  Gerard dreamed of a youthful
moment of sun and water and rolled away from Sophie.  Together they
each were whole in the sense of self, the natural character of the
{\it homo sapien} in which they rejoiced separately and together.

Gefou's theory of mind discarded the organic model developed since
Freud in favor of the anthropological observation of the {\it homo
sapien} as having a very small number of primary relationships which
determine its state of mind.  The first primary being the relationship
between the inner and outer consciousness or the relationship to the
self, the second primary being the relationship to a mate or partner,
and a third primary being the relationship with society or community
or world.  The relationship to children was a component of the partner
primary, and likewise particular relationships to groups and
communities were components of the world primary.  In this sense
psychology was the metaphysical source domain for alternative
scientific hypotheses, or as she preferred the innate facility to
capture the character of another.  With this theory the location of
the state of mind could be normal or abnormal with respect to the
architecture or expectations of society without being necessarily
unhealthy and requiring medical attention.  The theory fit well with
both anthropology and the tradition and experience of medicine.  It
permitted medical doctors to debate the wisdom of medication in the
context of the patient's expression of preference without denying the
existence of the consciousness of the patient as had resulted from the
organic model of mind that had delivered the superiority fantasy that
doctors had relied upon to manage the relationship to the subject of
their inspections.  The theory was physical and readily tested and
applied.  Gefou maintained great respect for metaphysics as the source
domain for scientific hypotheses, and metaphysicians as the producers
of that invaluable product, while asserting her objective and
subjective material with the clarity readers required to differentiate
her products from that of others.  Naturally as an academic he valued
each stitch in the fabric of knowledge and the communication of
knowledge.  The challenge that most readers faced was the place and
character of scientific certitude.  Gefou would not cheat the reader
of the scientific perspective, nor would she insult her work by
mistaking physics for metaphysics.  If it was her conjecture that the
field of psychology denied the self in a fantasy of superiority by the
separation of self from consciousness, then she wrote it plainly and
succinctly without qualification or disingenuity.  This was the
typical source of passion among students of her work, and a lesson in
its own right.  To differentiate physics and metaphysics and thereby
respect and maintain the substance and significance of each.  A heroic
worker on the tapestry of human knowledge.



\bye
